%%%% IACR Transactions TEMPLATE %%%%
% This file shows how to use the iacrtrans class to write a paper.
% Written by Gaetan Leurent gaetan.leurent@inria.fr (2020)
% Public Domain (CC0)


%%%% 1. DOCUMENTCLASS %%%%
\documentclass{iacrtrans}
%%%% NOTES:
% - Change "journal=tosc" to "journal=tches" if needed
% - Change "submission" to "final" for final version
% - Add "spthm" for LNCS-like theorems


%%%% 2. PACKAGES %%%%
\usepackage{lipsum} % Example package -- can be removed
\usepackage{booktabs}
\usepackage{mdframed}
\usepackage{amsmath,amsfonts,amssymb}
\usepackage{geometry}
\usepackage{enumitem}
\usepackage{hyperref}
\usepackage{longtable}
\usepackage{pdflscape}

\usepackage{framed}

\setcounter{MaxMatrixCols}{30}


%%% COMMANDS
\newcommand{\jbel}[1]{{\color{blue}{}jbel: #1}}
\newcommand{\ndhy}[2]{{\color{blue}{}ndhy: #2}}
\newcommand{\rand}{\overset{{\scriptscriptstyle\$}}{\leftarrow}}

%%%% 3. AUTHOR, INSTITUTE %%%%
%\author{Jane Doe\inst{1,2} \and John Doe\inst{1}}
%\institute{
%  Institute A, City, Country, \email{jane@institute}
%  \and
%  Institute B, City, Country, \email{john@institute}
%}
%%%% NOTES:
% - We need a city name for indexation purpose, even if it is redundant
%   (eg: University of Atlantis, Atlantis, Atlantis)
% - \inst{} can be omitted if there is a single institute,
%   or exactly one institute per author

\author{The zkID Team}
\institute{ Ethereum Foundation}

%%%% 4. TITLE %%%%
\title{Technical Overview: zkID for the EUDI Wallet}
%%%% NOTES:
% - If the title is too long, or includes special macro, please
%   provide a "running title" as optional argument: \title[Short]{Long}
% - You can provide an optional subtitle with \subtitle.

\begin{document}

\maketitle


%%%% 5. KEYWORDS %%%%
\keywords{Anonymous credential \and programmable zkp}


%%%% 6. ABSTRACT %%%%
%\begin{abstract}
%  Main deliveries: 1. Technical report on zk component for the digital id wallet 2. A comparison with current works 3. Applying to EUDI.
%\end{abstract}


%%%% 7. PAPER CONTENT %%%%
% \section{Introduction}
% \label{sec:introduction}
% Introducing DI and Selective disclosure. From Selective Disclosure to VCs lead to Anonymous credential.
% In Anonymous credential, talking about related works and their main approaches.
% \textit{In this section, we will present the definition of the scheme, the need for it, and our scope of work.}
% \input{intro.tex}
% \subsection{Problems and our scope of work}
% \begin{itemize}
%     \item What are we trying to solve, which pain points? 
%     \begin{itemize}
%         \item \textbf{\href{https://mirror.xyz/privacy-scaling-explorations.eth/zRM7qQSt_igfoSxdSa0Pts9MFdAoD96DD3m43bPQJT8}{This doc} dive deeper in the current problems and some current tool could be used \\ in future solutions}
%         \item \textbf{Answer described in this docs, \href{https://www.notion.so/pse-team/External-zkID-ZKP-Wallet-Unit-Proposal-1bad57e8dd7e80c98d73fc7e7666375d?pvs=25\#1bad57e8dd7e8059a446ca7b1dc31323}{Short-term Deliverables $\&$ Further Exploration}}
%         \item \textbf{And in this \href{https://pse.dev/en/projects/zk-id}{doc}} 
%     \end{itemize}
% \end{itemize}
% \subsection{Our achievements}
% \begin{itemize}
%     \item We need to describe zkID as simple as possible. What is it based on, like, well-known terms, and well-known algorithms?... \textit{I don't know the final zkID construction yet. Based on \href{https://pse-team.notion.site/zkID-Team-Strategy-Proposal-db3c5788dc7a4916a33e580a79177053}{\textbf{this proposal}}, I think there is a PoC construction, but I don't have access permission to it.}
%     \jbel{the full architecture and POC don't exist yet -- there are several streams going in parallel (some people thinking about architecture, some people benchmarking to inform that, some people creating a POC)}
%     \begin{itemize}
%         \item What are the main achievements, main results of our work?
%         \item What are the properties that our scheme satisfies?
%         \item What is the best experiment results and the environment of it?
%     \end{itemize} 
% \end{itemize}
% \subsection{Related work and how they make things right}
% \begin{itemize}
%     \item Before us, are there any solutions for these problems, and what are their pros and cons?
%     \begin{itemize}
%         \item \textbf{Overrall about current approaches is listed in \href{https://docs.google.com/presentation/d/1YROCEHK_t10wo5CukgYWmS1nuYKZi46NJBu-XZ8zXiw/edit?slide=id.p\#slide=id.p}{this presentation} (also in \href{https://docs.google.com/presentation/d/1HqFtSiS2hVHaSS8-u-8iecVFeMehMGBtZJnnbnXj83c/edit?slide=id.g34d4bb36836_0_262\#slide=id.g34d4bb36836_0_262}{this shorter version}), 
%  need to dig deeper to know how they solve these problems above.}
%         \item \textbf{\href{https://docs.google.com/presentation/d/1C4D8zK4gAdafgIEW-2m_qDyyT39gWo0mmFYpwmA8N3M/edit?slide=id.g312b09519cd_0_8\#slide=id.g312b09519cd_0_8}{This slide} also describe similar content but dive deeper in technical, and have the proposal design constraints.}
%         \begin{itemize}
%             \item Google solution: Tradeoffs and Considerations. 
%                 \begin{itemize}
%                     \item Pros: 
%                 \end{itemize}
                
%             \item Microsoft solution: Tradeoffs and Considerations.
%             \item Our zkID:
%                 * preprocessing 
%         \end{itemize}
%     \end{itemize} 
% \end{itemize}


\section{Introduction}
\label{sec:introduction}
%\input{intro}
According to the Cryptographers' Feedback on the EU Digital Identity’s ARF\footnote{\url{https://github.com/user-attachments/files/15904122/cryptographers-feedback.pdf}}, an Anonymous Credential AC scheme, is a suitable cryptographic primitive to instantiate the new EU Digital Identity Wallet (EUDIW) which is an important step towards developing interoperable digital identities in Europe for the public and private sectors.

\begin{framed}\footnotesize
	Informally speaking, an \emph{Anonymous Credential} AC scheme allows:
	\begin{itemize}
		\item An I\emph{dentity Provider} or \emph{Issuer} IP to (possibly blindly\footnote{i.e. the IP does not know the content that it signs, only its provenance is satisfied.}) sign a set of (eligible) attributes for a \emph{User} U;
		\item The \emph{User} U can show, only if they hold the signed attributes (a.k.a \emph{Unforgebality}), usually through a \emph{Presentation}, to a \emph{Relying Party} RP such that:
		\begin{itemize}
			\item The RP can verify that the set of attributes (signed by IP) that the User U holds satisfy some condition of their interest (a.k.a Correctness);
			\item The RP cannot learn any \emph{additional}\footnote{We stress that the RP may have obtained some privacy sensitive information prior to this presentation.} information beyond the fact that the condition is satisfied or information that can be inferred from the satisfaction of the condition (a.k.a Zero-Knowledge or Anonymity);
			\item The immediate previous requirement also implies that the RP cannot link the various presentations by the same User U (a.k.a. Unlinkability);
		\end{itemize}
		\item The IP can revoke all or a part of the signed attributes that it has issued to the User U, from upon which, the eligible attributes of the User U are updated, and subsequent presentations have to be based on the new and updated attributes (a.k.a \emph{Revocation});
		\item The User U cannot transfer its set of signed attributes to to another User U' (a.k.a \emph{Non-transferability}).
	\end{itemize}
\end{framed}


In the aforementioned feedback document, BBS and BBS+
%\footnote{For BBS, thanks to prior work by the W3C, the Decentralized Identity Foundation, IETF/IRTF, ISO, and other standardization bodies, as well as the availability of open-source software libraries, the EC can develop a standard and reference implementation with only a modest effort. The feedback additionally recommend that the EUDI be designed following the principle of crypto-agility, meaning that its underlying technologies can be upgraded quickly in the future if the need arises.} 
were promoted as the main candidate, besides that, there have been two independent work from Google and Microsoft that attempted to offer candidate solutions. In this document, we attempt to offer a new candidate, called \textbf{zkID}.

In comparison, these approaches show the current trade-off: systems either reuse existing issuer infrastructure but pay high per-presentation costs, or they achieve fast online proofs at the price of large setups and pairing-based assumptions. 
\begin{quote}
	\emph{Our construction, zkID, aims to combine issuer compatibility with reusable offline work, while remaining transparent and modular.}
\end{quote}


\subsection{Related Work}
\paragraph{Setting the AC Framework.}
Let us first outline a reference architecture that represents what an anonymous-credential system would ideally look like if it is to integrate smoothly with current infrastructures. 
In this model, the \emph{Issuer} is treated as fixed components that continue to use their existing public-key algorithms (such as RSA or ECDSA) and standard credential formats (e.g., JWT or mDL), since it's typically difficult to change once deployed. All additional logic is placed in the user’s wallet and the verifier.

The wallet is expected to operate in two stages: an offline \emph{Prepare} step, which verifies the Issuer’s signature once using standard libraries, parses and normalizes credential attributes (for example, turning a date of birth into an integer age), and commits to those attributes using a binding and hiding commitment scheme (a cryptographic way to lock values so they can later be revealed or proven in restricted form); and an online \emph{Show} step, which runs per presentation, where the wallet selects only the attributes or predicates required by a \emph{Relying Party}’s policy, proves them in zero knowledge against the stored commitments, and includes a fresh device signature over the session challenge to ensure the proof is tied to the holder’s device.

A further requirement is \emph{modularity}: each major function---issuer signature verification, attribute commitment, predicate proofs, and device binding---should be defined as a separate module with a clear interface. This separation makes it possible to swap the underlying proof engine (for example, using a SNARK today or a post-quantum proof system in the future) without requiring changes to parts of the system that are costly or impractical to modify. The purpose of this modular view is to act as a comparison framework: it outlines how a deployment-friendly anonymous-credential stack could be structured, making it easier to compare proposals by the modules they cover, the constraints they address, and the trade-offs they make.

\paragraph{BBS-based anonymous credentials.~\cite{baum2024cryptographers}}
BBS-based anonymous credentials are recommended in public feedback for the EUDI wallet as a way to meet the program’s requirement that presentations must not be tracked, linked, or correlated~\cite{baum2024cryptographers}.
This work treats a credential as a constant-size signature on a vector of attributes in pairing-friendly groups, as introduced by Boneh–Boyen–Shacham and proven secure for BBS+ by Au–Susilo–Mu~\cite{C:BonBoySha04,SCN:AuSusMu06}.
A holder then produces zero-knowledge proofs that reveal only the required attributes or predicates; each presentation is freshly generated so separate verifications cannot be linked.
This matches our reference system view on the presentation side-privacy enforced at the holder with per-session, non-repeating outputs.
Where these designs differ from our constraints is issuance. To use BBS/BBS+, issuers sign credentials with a pairing-based scheme rather than the RSA or ECDSA schemes used today~\cite{C:BonBoySha04,SCN:AuSusMu06}. To remain compatible with standardized curves such as P-256 while keeping public verifiability, a pairing-free, server-aided variant (often termed BBS\#) allows the holder to prefetch small auxiliary data through an oblivious interaction with an issuer-side helper and later perform non-interactive presentations; the helper data scales linearly with the number of planned presentations~\cite{cryptoeprint:2025/513}.
In both variants, device binding and revocation checks can be encoded as attributes or verified within the proof so that transcripts and status queries avoid stable identifiers.


\paragraph{Anonymous Credentials from ECDSA.~\cite{cryptoeprint:2024/2010}}
This work considers environments where credential issuers already sign with ECDSA on standardized curves (such as P-256) and hash data with SHA-256.
The main challenge is that proving correctness of an ECDSA signature in zero knowledge is costly with standard proof systems, because the arithmetic used in P-256 and the bit-level operations in SHA-256 do not align well with the fast polynomial techniques (such as number-theoretic transforms, a method that speeds up polynomial multiplication over special fields) that many modern ZK libraries rely on.
To handle this, the authors introduce custom circuits for ECDSA and SHA-256, and use a layered protocol based on the sum-check technique and a lightweight encoding (Reed–Solomon code) to control proof size.
An additional “consistency check” ensures that the same hidden signing key is used across both the signature and the hash logic.
At presentation, the wallet produces a proof for the verifier and the device also signs a fresh challenge (this is the device-binding step: a live signature that ties the proof to the holder’s device).
In terms of the reference system view, issuer compatibility is preserved, selective disclosure is supported, and device binding is included; however, there is no reusable offline phase, so the full proof is generated at every presentation. The reported costs are about 60\,ms to prove one ECDSA signature and about 1.2\,s for a complete mDL presentation on mobile devices~\cite[\S5.3,\S6.2]{cryptoeprint:2024/2010}, with larger proof sizes and higher verifier effort than systems based on succinct setup-dependent SNARKs.

\paragraph{Crescent Credentials.~\cite{cryptoeprint:2024/2013}}
This work considers environments where issuers continue using existing credential formats such as JWT or mDL and their current signing keys, so no issuer-side changes are required.
Its workflow is split into a heavy one-time Prepare phase and a lightweight per-presentation Show phase.
In Prepare, the wallet verifies the issuer’s signature, parses the credential into attributes, and creates two reusable artifacts---that is, cryptographic objects the wallet reuses across presentations: (i) a Groth16 proof that these checks were done correctly, and (ii) a Pedersen vector commitment over the attributes, enabling selective disclosure.
Both artifacts support re-randomization for unlinkability.
In the Show phase, the wallet re-randomizes the prepared artifacts and attaches only the proofs required by the verifier’s policy, such as proving an age threshold or linking two credentials to the same holder. Device binding can be added at this step by letting the secure element sign the verifier’s challenge.
In terms of the reference system view, Crescent realizes the two-phase design with reusable offline work and modular predicates, while leaving issuers unchanged. The trade-offs are significant: the Prepare phase is heavy (tens of seconds for JWTs and minutes for mDLs), the scheme depends on pairing-based Groth16 proofs with a large universal setup ($\approx$ 661 MB–1.1 GB~\cite[\S4]{cryptoeprint:2024/2013}), and the security model is classical only, without post-quantum protection. The Show step, however, runs with low latency-typically 22–41\,ms with $\approx$1 KB proofs, or about 315\,ms with device binding~\cite[\S4]{cryptoeprint:2024/2013}.

\subsection{Our zkID}

Our construction works with standardized credentials (e.g., SD-JWT, mDL) and existing PKI (RSA/ECDSA), so issuers do not need to change their issuance pipelines.
The zkID workflow follows the two-phase split in the reference view: a one-time Prepare phase and a per-presentation Show phase.
In Prepare, the wallet verifies the issuer’s signature, parses the credential into normalized messages, computes the associated hashes, and produces two reusable artifacts: (i) zero-knowledge proofs that issuer-side checks and parsing were done correctly, and (ii) Hyrax-style Pedersen vector commitments to a designated message column, supporting efficient proofs over multiple attributes.
In Show, the wallet proves only the verifier’s requested predicates and includes a fresh device-binding signature. To link Prepare and Show without revealing values, the verifier checks equality of commitments across both proofs; the wallet reuses the corresponding randomness for that session.
The proving backend is transparent (no trusted setup). It checks the arithmetic constraints with a sum-check–style protocol and uses a small inner-product check to verify commitment openings. For device binding, we choose a curve whose scalar field matches the device’s signature field (e.g., P-256), so the device signature can be verified directly inside the proof without emulation or field translation. 
In terms of the reference system view, issuer compatibility is preserved, the two-phase reuse is integrated into the workflow, predicates are modular, and there is no trusted setup. The trade-offs are that security currently relies on discrete-log assumptions (not post-quantum) and that commitment equality requires using the same curve across Prepare and Show; the modular interface leaves room to swap in lattice-based commitments when suitable.

\section{Application to EUDI}
\label{sec:appeudi}

\input{eudi_application}
% \jbel{Links to read (we can basically copy paste a lot from these):
% \begin{itemize}
% \item EUDI ARF (full) \href{https://eu-digital-identity-wallet.github.io/eudi-doc-architecture-and-reference-framework/latest/}{here}
% \item Discussion of Google/Microsoft pros/cons \href{https://github.com/eu-digital-identity-wallet/eudi-doc-standards-and-technical-specifications/blob/main/docs/technical-specifications/ts4-zkp.md}{here}
% \item Google's IETF draft for libZK \href{https://www.ietf.org/id/draft-google-cfrg-libzk-00.html#name-sumcheck}{here}
% \end{itemize}}

% Why should EUDI consider this report?
% \begin{itemize}
%     \item Does it compatible with current EUDI decision like data format and ecosystems?
%     \item Why governments or organizations should choose this scheme?
%     \item \textbf{Our pros and cons are already shown in other sections, so just mentioned them when we need them in this section}
% \end{itemize}

\section{Security}
\label{sec:security}

\input{security}

\section{Preliminaries - WIP}
\label{sec:preliminaries}
% \ndhy{This section is a work in progress. It will be completed in the next few days.}
\input{preliminaries}

% Give a detailed answer and analysis for:
% \begin{itemize}
%     \item Is the scheme a dishonest majority setting or something else? What happens when the setting is broken?
%     \jbel{\begin{itemize}
%         \item ZK is addressing malicious verifier -- semihonest 
%         \item soundness -- address malicious prover
%         \item trust assumption - verifier is trusted/honest, issuer is trusted/honest
%         \item deniable presentation? -- ask YT
%     \end{itemize}}
        
%     \item If the Issuer needs to update frequently, what if they are disconnected for a while? 
%     \item Place the scheme into a poor network connection, does it still work well and not be vulnerable?
%     \jbel{depends on solution to revocation flow, and also what applications of ID presentation look like (e.g. are the prover and verifier talking through internet channels?)}
% 	\item If it fails during the process, what will happen?
% 	\item If it is not quantum resistant, how do we upgrade it to quantum resistant? -- it is quantum resistant
% \end{itemize}

\section{Our zkID}
\label{sec:contribution}
% \textit{main deliveries: 1. describe zkID; 2. the detailed construction}
\input{zkID_construction.tex}

\section{Experiments}
\label{sec:experiments}
\input{experiments.tex}
% \begin{enumerate}
%     \item What are the exact communication, computation, and storage costs of \begin{itemize}
%         \item The PID Provider \& Verifier.
%         \item User who uses a mobile (iOS, Android) or browser?
%         \item The detailed cost of setup, proving, and verifying steps.
%     \end{itemize}
%     \item What is the minimum hardware configuration? 
% \end{enumerate}

% \textbf{We can extract the benchmark results from \href{https://hackmd.io/@clientsideproving/zkIDBenchmarks}{this doc} (it's \href{https://github.com/privacy-scaling-explorations/zkid-benchmarks}{git repo}) }

% \jbel{Still waiting 1-2 weeks for benchmarks on the POC}

% Notably, we also show the detailed comparison between our zkID and solutions from GG and Microsoft.

\section{Conclusion}
\label{sec:conclusion}

\clearpage
\section*{Contributors and Acknowledgement}
\begin{description}
	\item[Ying Tong] description
	\item[Hy Ngo]
	\item[Janabel]
\end{description}

We thanks X Y Z (Nam Zoey and everyone who said something or helped with something) for ...

\clearpage
%%%% 8. BILBIOGRAPHY %%%%
\bibliographystyle{alpha}
\bibliography{abbrev0,crypto,biblio,references}
%%%% NOTESr
% - Download abbrev3.bib and crypto.bib from https://cryptobib.di.ens.fr/
% - Use biblio.bib for additional references not in the cryptobib database.
%   If possible, take them from DBLP.

\clearpage
\section{Appendix: EUDI Annex 2 Requirements}
\label{sec:annex}

This section is devoted to a review of the EUDI ARF's Annex 2, which covers high-level requirements for the EUDI Wallet. 
The full Annex can be found \href{https://eu-digital-identity-wallet.github.io/eudi-doc-architecture-and-reference-framework/1.4.0/annexes/annex-2/annex-2-high-level-requirements/#a231-topic-1-accessing-public-and-private-online-services-with-eudi-wallet}{here}.

\addcontentsline{toc}{section}{Annex 2 – Mapping zkID to EUDI ARF Requirements}

\noindent The EUDI Annex~2 covers over fifty topics spanning cryptographic guarantees, privacy, trust infrastructure, wallet lifecycle management, and product-level usability. We cluster the requirements into four classes depending on their relationship to the zkID protocol itself:

\begin{enumerate}[label=\textbf{Group \Alph*:}]
  \item \textbf{Directly satisfied by zkID.} \\
  Requirements that are already implemented by the zkID protocol as described in Sections~\S5--\S5.3, without additional assumptions or infrastructure. These are shown in Table~A.

  \item \textbf{Satisfied with minor extensions.} \\
  Requirements that can be met by trivial modifications or configuration changes to zkID’s proving circuits, predicates, or interface (for instance, adding a new predicate or exposing one more public input). No new trust anchor or cryptographic primitive is required. These appear in Table~B.

  \item \textbf{Depend on external systems or operational flows.} \\
  Requirements that require the presence of registries, PKI governance, revocation lists, wallet attestation management, or other policy frameworks external to the proving layer. zkID integrates with these systems but does not replace them. These are summarized in Table~C.

  \item \textbf{Product / UX / policy-facing requirements.} \\
  Requirements that concern wallet behaviour, user experience, accessibility, or legal presentation. These fall outside the scope of cryptographic protocol design but are compatible with zkID’s guarantees. These are listed in Table~D.
\end{enumerate}

\medskip
\noindent
Each table records:
\begin{itemize}
  \item the corresponding Annex~2 topic and its HLRs;
  \item a short informal description of the requirement;
  \item and the relevant subsection(s) in this document where we discuss how zkID meets or relates to the requirement.
\end{itemize}

\medskip
\noindent
The following four landscape tables provide the detailed mapping for each group:

% Mapping tables 
% Group 1
\clearpage
\begin{landscape}
\small
\begin{longtable}{p{3cm} p{10cm} p{7cm}}
\caption*{Table A — Requirements directly implemented by zkID}\\
\toprule
\textbf{Annex 2 ID} &
\textbf{Requirement} &
\textbf{Where in zkID} \\
\midrule
\endfirsthead
\toprule
\textbf{Annex 2 ID} &
\textbf{Requirement} &
\textbf{Where in zkID} \\
\midrule
\endhead
\midrule
\multicolumn{3}{r}{\emph{continued on next page}}\\
\bottomrule
\endfoot
\bottomrule
\endlastfoot

\multicolumn{3}{l}{\textbf{Topic 1 — Online Identification and Authentication (OIA)}}\\

OIA\_01 &
For OIA\_01, see Interface in \S~5. &
see Interface in \S~5. \\

OIA\_02 &
For OIA\_02, see Component prepare batches in \S~5.2.1 and Component linking in \S~5.2. &
see component prepare batches in \S~5.2.1 and component linking in \S~5.2. \\

OIA\_03a &
For OIA\_03a, see Component predicate in \S~5.1 and \S~5.2.2 and Components zkSNARK wrapper in \S~5 and \S~5.1. &
see component predicate in \S~5.1 and \S~5.2.2 and zkSNARK wrapper in \S~5 and \S~5.1. \\

OIA\_03b &
For OIA\_03b, see Prepare relation in \S~5.2 and Show relation in \S~5.2. &
see Prepare relation in \S~5.2 and Show relation in \S~5.2. \\

OIA\_03c &
For OIA\_03c, see Prepare relation in \S~5.2 and Component predicate in \S~5.1 and \S~5.2.2. &
see Prepare relation in \S~5.2 and component predicate in \S~5.1 and \S~5.2.2. \\

OIA\_04 &
For OIA\_04, see Component predicate in \S~5.1 and \S~5.2.2 and Show relation in \S~5.2. &
see component predicate in \S~5.1 and \S~5.2.2 and Show relation in \S~5.2. \\

OIA\_05 &
For OIA\_05, see Component predicate in \S~5.1 and \S~5.2.2 and Security notes in \S~6. &
see component predicate in \S~5.1 and \S~5.2.2 and Security notes in \S~3. \\

OIA\_06 &
For OIA\_06, see Component predicate in \S~5.1 and \S~5.2.2. &
see component predicate in \S~5.1 and \S~5.2.2. \\

OIA\_07 &
For OIA\_07, see Component prepare batches in \S~5.2.1, Components zkSNARK wrapper in \S~5 and \S~5.1, and Component predicate in \S~5.1 and \S~5.2.2. &
see component prepare batches in \S~5.2.1, zkSNARK wrapper in \S~5 and \S~5.1, and component predicate in \S~5.1 and \S~5.2.2. \\

OIA\_08 &
For OIA\_08, see discussion in Security notes in \S~6 and Components zkSNARK wrapper in \S~5 and \S~5.1. &
see Security notes in \S~3 and zkSNARK wrapper in \S~5 and \S~5.1. \\

OIA\_09 &
For OIA\_09, see discussion in Security notes in \S~6 and Components zkSNARK wrapper in \S~5 and \S~5.1. &
see Security notes in \S~3 and zkSNARK wrapper in \S~5 and \S~5.1. \\

OIA\_10 &
For OIA\_10, see Component predicate in \S~5.1 and \S~5.2.2 and Show relation in \S~5.2. &
see component predicate in \S~5.1 and \S~5.2.2 and Show relation in \S~5.2. \\

OIA\_11 &
For OIA\_11, see Component predicate in \S~5.1 and \S~5.2.2 and Show relation in \S~5.2. &
see component predicate in \S~5.1 and \S~5.2.2 and Show relation in \S~5.2. \\

OIA\_12 &
For OIA\_12, see Prepare relation in \S~5.2.1 and discussion in Security notes in \S~6. &
see Prepare relation in \S~5.2.1 and Security notes in \S~3. \\

OIA\_13 &
For OIA\_13, see Prepare relation in \S~5.2.1 and discussion in Security notes in \S~6. &
see Prepare relation in \S~5.2.1 and Security notes in \S~3. \\

OIA\_14 &
For OIA\_14, see Prepare relation in \S~5.2.1 and discussion in Security notes in \S~6. &
see Prepare relation in \S~5.2.1 and Security notes in \S~3. \\

OIA\_15 &
For OIA\_15, see Prepare relation in \S~5.2.1 and discussion in Security notes in \S~6. &
see Prepare relation in \S~5.2.1 and Security notes in \S~3. \\

OIA\_16 &
For OIA\_16, see Security notes in \S~6 and Component predicate in \S~5.1 and \S~5.2.2. &
see Security notes in \S~3 and component predicate in \S~5.1 and \S~5.2.2. \\[1em]

\multicolumn{3}{l}{\textbf{Topic 10 — Issuance and Credential Handling (ISSU)}}\\

ISSU\_02 &
For ISSU\_02, see components SD-JWT/mDL wrapper in \S~5 and \S~5.1 and Components zkSNARK wrapper in \S~5 and \S~5.1. &
see SD-JWT/mDL wrapper in \S~5 and \S~5.1 and zkSNARK wrapper in \S~5 and \S~5.1. \\

ISSU\_07 &
For ISSU\_07, see components Prepare relation in \S~5.2.1 and prepareCommit in \S~5.2.1. &
see Prepare relation in \S~5.2.1 and \texttt{prepareCommit} in \S~5.2.1. \\

ISSU\_08 &
For ISSU\_08, see components Prepare relation in \S~5.2.1 and prepareCommit in \S~5.2.1. &
see Prepare relation in \S~5.2.1 and \texttt{prepareCommit} in \S~5.2.1. \\

ISSU\_09 &
For ISSU\_09, see components Prover's side discussion in \S~6 and Security notes in \S~6. &
see Prover-side discussion in \S~3 and Security notes in \S~3. \\

ISSU\_10 &
For ISSU\_10, see components Prepare relation in \S~5.2.1 and Prepare relation in \S~5.2. &
see Prepare relation in \S~5.2.1 and Prepare relation in \S~5.2. \\

ISSU\_12 &
For ISSU\_12, see components SD-JWT/mDL wrapper in \S~5 and \S~5.1 and Components zkSNARK wrapper in \S~5 and \S~5.1. &
see SD-JWT/mDL wrapper in \S~5 and \S~5.1 and zkSNARK wrapper in \S~5 and \S~5.1. \\

ISSU\_12a &
For ISSU\_12a, see components SD-JWT/mDL wrapper in \S~5 and \S~5.1 and Proof interface in \S~5.2. &
see SD-JWT/mDL wrapper in \S~5 and \S~5.1 and Proof interface in \S~5.2. \\

ISSU\_16 &
For ISSU\_16, see components SD-JWT/mDL wrapper in \S~5 and \S~5.1 and Interface in \S~5. &
see SD-JWT/mDL wrapper in \S~5 and \S~5.1 and Interface in \S~5. \\

ISSU\_27 &
For ISSU\_27, see components Show relation in \S~5.2.2 and Show relation in \S~5.2. &
see Show relation in \S~5.2.2 and Show relation in \S~5.2. \\

ISSU\_33 &
For ISSU\_33, see components SD-JWT/mDL wrapper in \S~5 and \S~5.1 and Component commitment in \S~5 and \S~5.2.1. &
see SD-JWT/mDL wrapper in \S~5 and \S~5.1 and component commitment in \S~5 and \S~5.2.1. \\

ISSU\_33a &
For ISSU\_33a, see components SD-JWT/mDL wrapper in \S~5 and \S~5.1 and Component commitment in \S~5 and \S~5.2.1. &
see SD-JWT/mDL wrapper in \S~5 and \S~5.1 and component commitment in \S~5 and \S~5.2.1. \\

ISSU\_33b &
For ISSU\_33b, see components prepareCommit in \S~5.2.1 and SD-JWT/mDL wrapper in \S~5 and \S~5.1. &
see \texttt{prepareCommit} in \S~5.2.1 and SD-JWT/mDL wrapper in \S~5 and \S~5.1. \\

ISSU\_35a &
For ISSU\_35a, see components prepareCommit in \S~5.2.1 and Prepare relation in \S~5.2. &
see \texttt{prepareCommit} in \S~5.2.1 and Prepare relation in \S~5.2. \\

ISSU\_37 &
For ISSU\_37, see components Component prepare batches in \S~5.2.1 and prepareBatch in \S~5.2.1. &
see component prepare batches in \S~5.2.1 and \texttt{prepareBatch} in \S~5.2.1. \\

ISSU\_37a &
For ISSU\_37a, see components prepareBatch in \S~5.2.1 and Proof interface in \S~5.2. &
see \texttt{prepareBatch} in \S~5.2.1 and Proof interface in \S~5.2. \\

ISSU\_38 &
For ISSU\_38, see components Show relation in \S~5.2.2 and Show relation in \S~5.2. &
see Show relation in \S~5.2.2 and Show relation in \S~5.2. \\

ISSU\_39 &
For ISSU\_39, see components prepareBatch in \S~5.2.1 and Prepare relation in \S~5.2. &
see \texttt{prepareBatch} in \S~5.2.1 and Prepare relation in \S~5.2. \\

ISSU\_41a &
For ISSU\_41a, see components Show relation in \S~5.2.2 and Show relation in \S~5.2. &
see Show relation in \S~5.2.2 and Show relation in \S~5.2. \\

ISSU\_41b &
For ISSU\_41b, see components Component prepare batches in \S~5.2.1 and Show relation in \S~5.2. &
see component prepare batches in \S~5.2.1 and Show relation in \S~5.2. \\

ISSU\_41c &
For ISSU\_41c, see components Show relation in \S~5.2.2 and Show relation in \S~5.2. &
see Show relation in \S~5.2.2 and Show relation in \S~5.2. \\

ISSU\_44 &
For ISSU\_44, see components Component prepare batches in \S~5.2.1 and Show relation in \S~5.2. &
see component prepare batches in \S~5.2.1 and Show relation in \S~5.2. \\

ISSU\_58 &
For ISSU\_58, see components Show relation in \S~5.2.2 and Show relation in \S~5.2. &
see Show relation in \S~5.2.2 and Show relation in \S~5.2. \\

ISSU\_59 &
For ISSU\_59, see components Show relation in \S~5.2.2 and Show relation in \S~5.2. &
see Show relation in \S~5.2.2 and Show relation in \S~5.2. \\

ISSU\_60 &
For ISSU\_60, see components Component predicate in \S~5.1 and \S~5.2.2 and Proof interface in \S~5.2. &
see component predicate in \S~5.1 and \S~5.2.2 and Proof interface in \S~5.2. \\

ISSU\_61 &
For ISSU\_61, see components Component predicate in \S~5.1 and \S~5.2.2 and Proof interface in \S~5.2. &
see component predicate in \S~5.1 and \S~5.2.2 and Proof interface in \S~5.2. \\

ISSU\_62 &
For ISSU\_62, see components SD-JWT/mDL wrapper in \S~5 and \S~5.1 and Components zkSNARK wrapper in \S~5 and \S~5.1. &
see SD-JWT/mDL wrapper in \S~5 and \S~5.1 and zkSNARK wrapper in \S~5 and \S~5.1. \\

ISSU\_63 &
For ISSU\_63, see components prepareCommit in \S~5.2.1 and SD-JWT/mDL wrapper in \S~5 and \S~5.1. &
see \texttt{prepareCommit} in \S~5.2.1 and SD-JWT/mDL wrapper in \S~5 and \S~5.1. \\

ISSU\_64 &
For ISSU\_64, see components Proof interface in \S~5.2 and Proof interface in \S~5.2. &
see Proof interface in \S~5.2. \\[1em]

\multicolumn{3}{l}{\textbf{Topic 23 — PID and (Q)EAA issuance}}\\

Topic 23 &
23 PID issuance and (Q)EAA issuance. No HLRs, see Topic 10. &
covered by Topic 10 components in \S~5, \S~5.1, \S~5.2.1, \S~5.2.2. \\[1em]

\multicolumn{3}{l}{\textbf{Topic 47 — Protocols and interfaces for PID and (Q)EAA issuance}}\\

Topic 47 &
47 Protocols and interfaces for PID and (Q)EAA issuance and (non-)qualified. No HLRs, see Topic 10, 23. &
covered by Topic 10 / Topic 23 components in \S~5, \S~5.1, \S~5.2.1, \S~5.2.2. \\[1em]

\multicolumn{3}{l}{\textbf{Topic 53 — Zero-Knowledge Proofs (ZKP)}}\\

ZKP\_01 &
For ZKP\_01, see components Component predicate in \S~5.1 and \S~5.2.2 and Security notes in \S~6. &
see component predicate in \S~5.1 and \S~5.2.2 and Security notes in \S~3. \\

ZKP\_02 &
For ZKP\_02, see components Prepare relation in \S~5.2 and Component predicate in \S~5.1 and \S~5.2.2. &
see Prepare relation in \S~5.2 and component predicate in \S~5.1 and \S~5.2.2. \\

ZKP\_03 &
For ZKP\_03, see components Component commitment in \S~5 and \S~5.2.1 and Component linking in \S~5.2. &
see component commitment in \S~5 and \S~5.2.1 and component linking in \S~5.2. \\

ZKP\_04 &
For ZKP\_04, see components Component predicate in \S~5.1 and \S~5.2.2 and Show relation in \S~5.2. &
see component predicate in \S~5.1 and \S~5.2.2 and Show relation in \S~5.2. \\

ZKP\_05 &
For ZKP\_05, see components Prepare relation in \S~5.2 and Show relation in \S~5.2. &
see Prepare relation in \S~5.2 and Show relation in \S~5.2. \\

ZKP\_06 &
For ZKP\_06, see components Components zkSNARK wrapper in \S~5 and \S~5.1 and SD-JWT/mDL wrapper in \S~5 and \S~5.1. &
see zkSNARK wrapper in \S~5 and \S~5.1 and SD-JWT/mDL wrapper in \S~5 and \S~5.1. \\

ZKP\_07 &
For ZKP\_07, see components Prepare relation in \S~5.2.1 and Component commitment in \S~5 and \S~5.2.1 in Security notes in \S~6 and \S~5. &
see Prepare relation in \S~5.2.1 and component commitment in \S~5 and \S~5.2.1 and Security notes in \S~3. \\

ZKP\_08 &
For ZKP\_08, see components Backend modularity in \S~5.3 and Security notes in \S~6. &
see backend modularity in \S~5.3 and Security notes in \S~3. \\

ZKP\_09 &
For ZKP\_09, see components Show relation in \S~5.2.2 and Proof interface in \S~5.2. &
see Show relation in \S~5.2.2 and Proof interface in \S~5.2. \\

\end{longtable}
\end{landscape}





% Group 2
\clearpage
\begin{landscape}
\small
\begin{longtable}{p{3cm} p{10cm} p{7cm}}
\caption*{Table B — Requirements implementable by minor extension or modification of zkID components}\\
\toprule
\textbf{Annex 2 ID} &
\textbf{Requirement} &
\textbf{Coverage} \\
\midrule
\endfirsthead
\toprule
\textbf{Annex 2 ID} &
\textbf{Requirement} &
\textbf{Coverage} \\
\midrule
\endhead
\midrule
\multicolumn{3}{r}{\emph{continued on next page}}\\
\bottomrule
\endfoot
\bottomrule
\endlastfoot

\multicolumn{3}{l}{\textbf{Topic 6 — Relying Party authentication and User approval}}\\

RPA\_01 &
For RPA\_01, modify components Prepare relation in \S~5.2.1 and Prover's side discussion in \S~6. &
modify components Prepare relation in \S~5.2.1 and Prover's side discussion in \S~3. \\

RPA\_01a &
For RPA\_01a, modify components Show relation in \S~5.2 and Proof interface in \S~5.2. &
modify components Show relation in \S~5.2 and Proof interface in \S~5.2. \\

RPA\_02 &
For RPA\_02, modify components Show relation in \S~5.2 and Component predicate in \S~5.1 and \S~5.2.2. &
modify components Show relation in \S~5.2 and Component predicate in \S~5.1 and \S~5.2.2. \\

RPA\_02a &
For RPA\_02a, modify components Proof interface in \S~5.2 and Component linking in \S~5.2. &
modify components Proof interface in \S~5.2 and Component linking in \S~5.2. \\

RPA\_03 &
For RPA\_03, modify components Prepare relation in \S~5.2.1 and Show relation in \S~5.2.2. &
modify components Prepare relation in \S~5.2.1 and Show relation in \S~5.2.2. \\

RPA\_04 &
For RPA\_04, modify components Prover's side discussion in \S~6 and Security notes in \S~6. &
modify components Prover's side discussion in \S~3 and Security notes in \S~3. \\

RPA\_05 &
For RPA\_05, modify components Proof interface in \S~5.2 and Component linking in \S~5.2. &
modify components Proof interface in \S~5.2 and Component linking in \S~5.2. \\

RPA\_06 &
For RPA\_06, modify components Component predicate in \S~5.1 and \S~5.2.2 and Show relation in \S~5.2. &
modify components Component predicate in \S~5.1 and \S~5.2.2 and Show relation in \S~5.2. \\

RPA\_06a &
For RPA\_06a, modify components Show relation in \S~5.2 and Proof interface in \S~5.2. &
modify components Show relation in \S~5.2 and Proof interface in \S~5.2. \\

RPA\_07 &
For RPA\_07, modify components Show relation in \S~5.2.2 and Component predicate in \S~5.1 and \S~5.2.2. &
modify components Show relation in \S~5.2.2 and Component predicate in \S~5.1 and \S~5.2.2. \\

RPA\_07a &
For RPA\_07a, modify components Show relation in \S~5.2 and Proof interface in \S~5.2. &
modify components Show relation in \S~5.2 and Proof interface in \S~5.2. \\

RPA\_08 &
For RPA\_08, modify components Show relation in \S~5.2.2 and Show relation in \S~5.2. &
modify components Show relation in \S~5.2.2 and Show relation in \S~5.2. \\

RPA\_09 &
For RPA\_09, modify components Component predicate in \S~5.1 and \S~5.2.2 and Show relation in \S~5.2. &
modify components Component predicate in \S~5.1 and \S~5.2.2 and Show relation in \S~5.2. \\

RPA\_10 &
For RPA\_10, modify components Proof interface in \S~5.2 and Proof interface in \S~5.2. &
modify components Proof interface in \S~5.2 and Proof interface in \S~5.2. \\[1em]


\multicolumn{3}{l}{\textbf{Topic 11 — Pseudonyms}}\\

PA\_01 &
For PA\_01, modify components Component predicate in \S~5.1 and \S~5.2.2 and Prepare relation in \S~5.2. &
modify components Component predicate in \S~5.1 and \S~5.2.2 and Prepare relation in \S~5.2. \\

PA\_02 &
For PA\_02, modify components Show relation in \S~5.2 and Show relation in \S~5.2.2. &
modify components Show relation in \S~5.2 and Show relation in \S~5.2.2. \\

PA\_03 &
For PA\_03, modify components Proof interface in \S~5.2 and Proof interface in \S~5.2. &
modify components Proof interface in \S~5.2 and Proof interface in \S~5.2. \\

PA\_04 &
For PA\_04, modify components Prepare relation in \S~5.2 and Component predicate in \S~5.1 and \S~5.2.2. &
modify components Prepare relation in \S~5.2 and Component predicate in \S~5.1 and \S~5.2.2. \\

PA\_05 &
For PA\_05, modify components Proof interface in \S~5.2 and Proof interface in \S~5.2. &
modify components Proof interface in \S~5.2 and Proof interface in \S~5.2. \\

PA\_06 &
For PA\_06, modify components Show relation in \S~5.2 and Proof interface in \S~5.2. &
modify components Show relation in \S~5.2 and Proof interface in \S~5.2. \\

PA\_07 &
For PA\_07, modify components Proof interface in \S~5.2 and Proof interface in \S~5.2. &
modify components Proof interface in \S~5.2 and Proof interface in \S~5.2. \\

PA\_08 &
For PA\_08, modify components Proof interface in \S~5.2 and Proof interface in \S~5.2. &
modify components Proof interface in \S~5.2 and Proof interface in \S~5.2. \\

PA\_08a &
For PA\_08a, modify components Security notes in \S~6 and Proof interface in \S~5.2. &
modify components Security notes in \S~3 and Proof interface in \S~5.2. \\

PA\_09 &
For PA\_09, modify components Proof interface in \S~5.2 and Proof interface in \S~5.2. &
modify components Proof interface in \S~5.2 and Proof interface in \S~5.2. \\

PA\_10 &
For PA\_10, modify components Show relation in \S~5.2.2 and Security notes in \S~6. &
modify components Show relation in \S~5.2.2 and Security notes in \S~3. \\

PA\_11 &
For PA\_11, modify components Show relation in \S~5.2.2 and Security notes in \S~6. &
modify components Show relation in \S~5.2.2 and Security notes in \S~3. \\

PA\_12 &
For PA\_12, modify components Show relation in \S~5.2.2 and Show relation in \S~5.2.2. &
modify components Show relation in \S~5.2.2 and Show relation in \S~5.2.2. \\

PA\_13 &
For PA\_13, modify components Show relation in \S~5.2.2 and Proof interface in \S~5.2. &
modify components Show relation in \S~5.2.2 and Proof interface in \S~5.2. \\

PA\_14 &
For PA\_14, modify components Show relation in \S~5.2.2 and Security notes in \S~6. &
modify components Show relation in \S~5.2.2 and Security notes in \S~3. \\

PA\_15 &
For PA\_15, modify components Prepare relation in \S~5.2.1 and Security notes in \S~6. &
modify components Prepare relation in \S~5.2.1 and Security notes in \S~3. \\

PA\_16 &
For PA\_16, modify components Prepare relation in \S~5.2.1 and Component predicate in \S~5.1 and \S~5.2.2. &
modify components Prepare relation in \S~5.2.1 and Component predicate in \S~5.1 and \S~5.2.2. \\

PA\_17 &
For PA\_17, modify components Component prepare batches in \S~5.2.1 and Prepare relation in \S~5.2.1. &
modify components Component prepare batches in \S~5.2.1 and Prepare relation in \S~5.2.1. \\

PA\_18 &
For PA\_18, modify components prepareCommit in \S~5.2.1 and Security notes in \S~6. &
modify components \texttt{prepareCommit} in \S~5.2.1 and Security notes in \S~3. \\

PA\_19 &
For PA\_19, modify components Proof interface in \S~5.2 and Security notes in \S~6. &
modify components Proof interface in \S~5.2 and Security notes in \S~3. \\[1em]


\multicolumn{3}{l}{\textbf{Topic 17 — Identity matching}}\\

No HLRs &
Enable users to access existing online accounts or log in to cross-border public sector services using their PID via identity matching, even if PID attribute values do not exactly match those in the accounts. &
modify components Prepare relation in \S~5.2.1 and Security notes in \S~3. \\[2em]


\multicolumn{3}{l}{\textbf{Topic 18 — Combined presentations of attributes}}\\

ACP\_01 &
For ACP\_01, modify components Component predicate in \S~5.1 and \S~5.2.2 and Component linking in \S~5.2. &
modify components Component predicate in \S~5.1 and \S~5.2.2 and Component linking in \S~5.2. \\

ACP\_02 &
For ACP\_02, modify components Component commitment in \S~5 and \S~5.2.1 and Show relation in \S~5.2.2. &
modify components Component commitment in \S~5 and \S~5.2.1 and Show relation in \S~5.2.2. \\

ACP\_03 &
For ACP\_03, modify components Prepare relation in \S~5.2 and Show relation in \S~5.2. &
modify components Prepare relation in \S~5.2 and Show relation in \S~5.2. \\

ACP\_04 &
For ACP\_04, modify components Proof interface in \S~5.2 and Proof interface in \S~5.2. &
modify components Proof interface in \S~5.2 and Proof interface in \S~5.2. \\

ACP\_05 &
For ACP\_05, modify components Component commitment in \S~5 and Component predicate in \S~5.1 and \S~5.2.2. &
modify components Component commitment in \S~5 and Component predicate in \S~5.1 and \S~5.2.2. \\

ACP\_06 &
For ACP\_06, modify components prepareCommit in \S~5.2.1 and prepareBatch in \S~5.2.1. &
modify components \texttt{prepareCommit} in \S~5.2.1 and \texttt{prepareBatch} in \S~5.2.1. \\

ACP\_07 &
For ACP\_07, modify components Prepare relation in \S~5.2.1 and Component prepare batches in \S~5.2.1. &
modify components Prepare relation in \S~5.2.1 and Component prepare batches in \S~5.2.1. \\[1em]


\multicolumn{3}{l}{\textbf{Topic 20 — Strong User authentication for electronic payments}}\\

SUA\_01 &
For SUA\_01, modify components Show relation in \S~5.2.2 and Show relation in \S~5.2. &
modify components Show relation in \S~5.2.2 and Show relation in \S~5.2. \\

SUA\_02 &
For SUA\_02, modify components Component predicate in \S~5.1 and \S~5.2.2 and Show relation in \S~5.2. &
modify components Component predicate in \S~5.1 and \S~5.2.2 and Show relation in \S~5.2. \\

SUA\_03 &
For SUA\_03, modify components Show relation in \S~5.2 and Prepare relation in \S~5.2. &
modify components Show relation in \S~5.2 and Prepare relation in \S~5.2. \\

SUA\_04 &
For SUA\_04, modify components Show relation in \S~5.2.2 and Component predicate in \S~5.1 and \S~5.2.2. &
modify components Show relation in \S~5.2.2 and Component predicate in \S~5.1 and \S~5.2.2. \\

SUA\_05 &
For SUA\_05, modify components Security notes in \S~6 and Show relation in \S~5.2. &
modify components Security notes in \S~3 and Show relation in \S~5.2. \\

SUA\_06 &
For SUA\_06, modify components Component predicate in \S~5.1 and \S~5.2.2 and Proof interface in \S~5.2. &
modify components Component predicate in \S~5.1 and \S~5.2.2 and Proof interface in \S~5.2. \\[1em]


\multicolumn{3}{l}{\textbf{Topic 43 — Embedded disclosure policies}}\\

EDP\_01 &
For EDP\_01, modify components Component commitment in \S~5 and \S~5.2.1 and prepareCommit in \S~5.2.1. &
modify components Component commitment in \S~5 and \S~5.2.1 and \texttt{prepareCommit} in \S~5.2.1. \\

EDP\_02 &
For EDP\_02, modify components Component predicate in \S~5.1 and \S~5.2.2 and Proof interface in \S~5.2. &
modify components Component predicate in \S~5.1 and \S~5.2.2 and Proof interface in \S~5.2. \\

EDP\_03 &
For EDP\_03, modify components Prover's side discussion in \S~6 and Security notes in \S~6. &
modify components Prover's side discussion in \S~3 and Security notes in \S~3. \\

EDP\_05 &
For EDP\_05, modify components Proof interface in \S~5.2 and Proof interface in \S~5.2. &
modify components Proof interface in \S~5.2 and Proof interface in \S~5.2. \\

EDP\_06 &
For EDP\_06, modify components Component predicate in \S~5.1 and \S~5.2.2 and Show relation in \S~5.2. &
modify components Component predicate in \S~5.1 and \S~5.2.2 and Show relation in \S~5.2. \\

EDP\_07 &
For EDP\_07, modify components Component predicate in \S~5.1 and \S~5.2.2 and Show relation in \S~5.2. &
modify components Component predicate in \S~5.1 and \S~5.2.2 and Show relation in \S~5.2. \\

EDP\_09 &
For EDP\_09, modify components prepareCommit in \S~5.2.1 and SD-JWT/mDL wrapper in \S~5 and \S~5.1. &
modify components \texttt{prepareCommit} in \S~5.2.1 and SD-JWT/mDL wrapper in \S~5 and \S~5.1. \\

EDP\_10 &
For EDP\_10, modify components prepareCommit in \S~5.2.1 and Proof interface in \S~5.2. &
modify components \texttt{prepareCommit} in \S~5.2.1 and Proof interface in \S~5.2. \\

EDP\_11 &
For EDP\_11, modify components Security notes in \S~6 and Prepare relation in \S~5.2. &
modify components Security notes in \S~3 and Prepare relation in \S~5.2. \\[1em]


\multicolumn{3}{l}{\textbf{Topic 51 — PID or attestation deletion}}\\

PAD\_01 &
For PAD\_01, modify components Proof interface in \S~5.2 and Proof interface in \S~5.2. &
modify components Proof interface in \S~5.2 and Proof interface in \S~5.2. \\

PAD\_02 &
For PAD\_02, modify components prepareCommit in \S~5.2.1 and SD-JWT/mDL wrapper in \S~5 and \S~5.1. &
modify components \texttt{prepareCommit} in \S~5.2.1 and SD-JWT/mDL wrapper in \S~5 and \S~5.1. \\

PAD\_03 &
For PAD\_03, modify components Component predicate in \S~5.1 and \S~5.2.2 and Show relation in \S~5.2. &
modify components Component predicate in \S~5.1 and \S~5.2.2 and Show relation in \S~5.2. \\

PAD\_04 &
For PAD\_04, modify components Show relation in \S~5.2.2 and Show relation in \S~5.2. &
modify components Show relation in \S~5.2.2 and Show relation in \S~5.2. \\

PAD\_05 &
For PAD\_05, modify components Security notes in \S~6 and Component commitment in \S~5 and \S~5.2.1. &
modify components Security notes in \S~3 and Component commitment in \S~5 and \S~5.2.1. \\

PAD\_06 &
For PAD\_06, modify components Component prepare batches in \S~5.2.1 and Show relation in \S~5.2. &
modify components Component prepare batches in \S~5.2.1 and Show relation in \S~5.2. \\[1em]


\multicolumn{3}{l}{\textbf{Topic 52 — Relying Party intermediaries}}\\

RPI\_01 & 
For RPI\_01, modify components Proof interface in S5.2 and Interface in S3. &
modify components Proof interface in \S~5.2 and Interface in \S~3. \\

RPI\_02 &
For RPI\_02, N/A (empty). &
N/A. \\

RPI\_03 &
For RPI\_03, modify components Proof interface in S5.2 and Proof interface in S5.2. &
modify components Proof interface in \S~5.2 and Proof interface in \S~5.2. \\

RPI\_04 &
For RPI\_04, modify components Prepare relation in S5.2.1 and Security notes in S3. &
modify components Prepare relation in \S~5.2.1 and Security notes in \S~3. \\

RPI\_05 &
For RPI\_05, modify components Proof interface in S5.2 and Proof interface in S5.2. &
modify components Proof interface in \S~5.2 and Proof interface in \S~5.2. \\

RPI\_06 &
For RPI\_06, modify components Proof interface in S5.2 and Show relation in S5.2. &
modify components Proof interface in \S~5.2 and Show relation in \S~5.2. \\

RPI\_06a &
For RPI\_06a, modify components Proof interface in S5.2 and SD-JWT/mDL wrapper in S3 and S5.1. &
modify components Proof interface in \S~5.2 and SD-JWT/mDL wrapper in \S~3 and \S~5.1. \\

RPI\_07 &
For RPI\_07, modify components Proof interface in S5.2 and Show relation in S5.2. &
modify components Proof interface in \S~5.2 and Show relation in \S~5.2. \\

RPI\_07a &
For RPI\_07a, modify components Prepare relation in S5.2.1 and Show relation in S5.2.2. &
modify components Prepare relation in \S~5.2.1 and Show relation in \S~5.2.2. \\

RPI\_07b &
For RPI\_07b, modify components Interface in S3 and Show relation in S5.2.2. &
modify components Interface in \S~3 and Show relation in \S~5.2.2. \\

RPI\_08 &
For RPI\_08, modify components Proof interface in S5.2 and Security notes in S3. &
modify components Proof interface in \S~5.2 and Security notes in \S~3. \\

RPI\_09 &
For RPI\_09, modify components Prepare relation in S5.2.1 and Component linking in S5.2. &
modify components Prepare relation in \S~5.2.1 and Component linking in \S~5.2. \\

RPI\_10 &
For RPI\_10, modify components Security notes in S3 and Proof interface in S5.2. &
modify components Security notes in \S~3 and Proof interface in \S~5.2. \\

\end{longtable}
\end{landscape}



% Group 3
\clearpage
\begin{landscape}
\small
\begin{longtable}{p{3cm} p{10cm} p{7cm}}
\caption*{Table C — Requirements requiring integration with external systems or protocol adaptations}\\
\toprule
\textbf{Annex 2 Topic / ID} &
\textbf{Requirement} &
\textbf{Coverage} \\
\midrule
\endfirsthead
\toprule
\textbf{Annex 2 Topic / ID} &
\textbf{Requirement} &
\textbf{Coverage} \\
\midrule
\endhead
\midrule
\multicolumn{3}{r}{\emph{continued on next page}}\\
\bottomrule
\endfoot
\bottomrule
\endlastfoot

\multicolumn{3}{l}{\textbf{Topic 2 — Mobile Driving Licence (mDL) within the EUDI Wallet ecosystem}}\\

Topic 2 &
 &
\\

\multicolumn{3}{l}{\textbf{Topic 3 — PID Rulebook}}\\

Topic 3 &
 &
\\

\multicolumn{3}{l}{\textbf{Topic 4 — mDL Rulebook}}\\

Topic 4 &
 &
\\

\multicolumn{3}{l}{\textbf{Topic 7 — Attestation revocation and revocation checking}}\\

Topic 7 &
 &
\\

\multicolumn{3}{l}{\textbf{Topic 9 — Wallet Instance Attestation / Wallet Unit Attestation}}\\

Topic 9 &
 &
\\

\multicolumn{3}{l}{\textbf{Topic 16 — Signing documents with a Wallet Unit}}\\

Topic 16 &
 &
\\

\multicolumn{3}{l}{\textbf{Topic 24 — User identification in proximity scenarios}}\\

Topic 24 &
 &
\\

\multicolumn{3}{l}{\textbf{Topic 25 — Unified definition and controlled vocabularies for attributes}}\\

Topic 25 &
 &
\\

\multicolumn{3}{l}{\textbf{Topic 26 — Catalogue of attestations}}\\

Topic 26 &
 &
\\

\multicolumn{3}{l}{\textbf{Topic 27 — Registration of PID Providers, Providers of QEAAs, PuB-EAAs, and non-qualified}}\\

Topic 27 &
 &
\\

\multicolumn{3}{l}{\textbf{Topic 30 — Interaction between Wallet Units}}\\

Topic 30 &
 &
\\

\multicolumn{3}{l}{\textbf{Topic 31 — Notification and publication of PID Provider / Wallet Provider / Attestation Provider trust status}}\\

Topic 31 &
 &
\\

\multicolumn{3}{l}{\textbf{Topic 33 — Wallet Unit backup and restore}}\\

Topic 33 &
 &
\\

\multicolumn{3}{l}{\textbf{Topic 35 — PID issuance service blueprint}}\\

Topic 35 &
 &
\\

\multicolumn{3}{l}{\textbf{Topic 37 — QES / Remote Signing — Technical Requirements}}\\

Topic 37 &
 &
\\

\multicolumn{3}{l}{\textbf{Topic 38 — Wallet Unit revocation}}\\

Topic 38 &
 &
\\

\multicolumn{3}{l}{\textbf{Topic 39 — Wallet-to-wallet technical topic}}\\

Topic 39 &
 &
\\

\multicolumn{3}{l}{\textbf{Topic 40 — Wallet Instance installation / activation / management}}\\

Topic 40 &
 &
\\

\multicolumn{3}{l}{\textbf{Topic 44 — Registration certificates for PID Providers, QEAAs, PuB-EAAs}}\\

Topic 44 &
 &
\\

\multicolumn{3}{l}{\textbf{Topic 48 — Blueprint for requesting data deletion to Relying Parties}}\\

Topic 48 &
 &
\\

\end{longtable}
\end{landscape}



% Group 4
\clearpage
\begin{landscape}
\small
\begin{longtable}{p{3cm} p{10cm} p{7cm}}
\caption*{Table D — Requirements dependent on product, user interface, or user-experience considerations}\\
\toprule
\textbf{Annex 2 Topic / ID} & \textbf{Requirement} & \textbf{Coverage}\\
\midrule
\endfirsthead
\toprule
\textbf{Annex 2 Topic / ID} & \textbf{Requirement} & \textbf{Coverage}\\
\midrule
\endhead
\midrule
\multicolumn{3}{r}{\emph{continued on next page}}\\
\bottomrule
\endfoot
\bottomrule
\endlastfoot

Topics 5, 8, 13, 14, 15, 21, 22, 36 &
There are no HLRs for this Topic. & \\

Topics 12, 32, 41, 45, 46 &
 & Refers to the attestation rulebook \\

Topics 24, 30 &
 & Refers to wallet instance requirements \\

Topic 28 &
 & Refers to PID rulebook \\

Topic 29 &
 & Refers to natural person PID \\

Topic 33 &
 & Refers to functional requirements of back up and restore function of wallet instance \\

Topic 42 &
... & ... \\

Topic 50 &
& Refers to compliance requirements of the wallet provider and wallet instance \\

Topic 54 &
... & ... \\

\end{longtable}
\end{landscape}


\end{document}
