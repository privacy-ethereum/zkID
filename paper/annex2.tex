% \addcontentsline{toc}{section}{Annex 2 – Mapping zkID to EUDI ARF Requirements}
\noindent The EUDI Annex~2 covers over fifty topics spanning cryptographic guarantees, privacy, trust infrastructure, wallet lifecycle management, and product-level usability. We cluster the requirements into four classes depending on their relationship to the zkID protocol itself:

\begin{description}
  \item[Group A] \textbf{Directly satisfied by zkID.} \\
  Requirements that are already implemented by the zkID protocol as described in Sections~\S5--\S5.3, without additional assumptions or infrastructure. These are shown in Table~\ref{tab:groupA}.

  \item[Group B] \textbf{Satisfied with minor extensions.} \\
  Requirements that can be met by trivial modifications or configuration changes to zkID’s proving circuits, predicates, or interface (for instance, adding a new predicate or exposing one more public input). No new trust anchor or cryptographic primitive is required. These appear in Table~\ref{tab:groupB}.

  \item[Group C] \textbf{Depend on external systems or operational flows.} \\
  Requirements that require the presence of registries, PKI governance, revocation lists, wallet attestation management, or other policy frameworks external to the proving layer. zkID integrates with these systems but does not replace them. These are summarized in Table~\ref{tab:groupC}.

  \item[Group D] \textbf{Product / UX / policy-facing requirements.} \\
  Requirements that concern wallet behaviour, user experience, accessibility, or legal presentation. These fall outside the scope of cryptographic protocol design but are compatible with zkID’s guarantees. These are listed in Table~\ref{tab:groupD}.
\end{description}

\medskip
\noindent Each table records:
\begin{itemize}
  \item the corresponding Annex~2 topic and its HLRs;
  \item a short informal description of the requirement; and
  \item the relevant subsection(s) in this document where we discuss how zkID meets or relates to the requirement.
\end{itemize}

\medskip
\noindent The following four landscape tables provide the detailed mapping for each group:
% Mapping tables 
% Group 1
\clearpage
\begin{landscape}
\small
\begin{longtable}{p{3cm} p{10cm} p{7cm}}
\caption*{Requirements directly implemented by zkID}\label{tab:groupA}\\
\toprule
\textbf{Annex 2 ID} &
\textbf{Requirement} &
\textbf{Where in zkID} \\
\midrule
\endfirsthead
\toprule
\textbf{Annex 2 ID} &
\textbf{Requirement} &
\textbf{Where in zkID} \\
\midrule
\endhead
\midrule
\multicolumn{3}{r}{\emph{continued on next page}}\\
\bottomrule
\endfoot
\bottomrule
\endlastfoot

\multicolumn{3}{l}{\textbf{Topic 1 — Online Identification and Authentication (OIA)}}\\

OIA\_01 &
For OIA\_01, see Interface in \S~5. &
see Interface in \S~5. \\

OIA\_02 &
For OIA\_02, see Component prepare batches in \S~5.2.1 and Component linking in \S~5.2. &
see component prepare batches in \S~5.2.1 and component linking in \S~5.2. \\

OIA\_03a &
For OIA\_03a, see Component predicate in \S~5.1 and \S~5.2.2 and Components zkSNARK wrapper in \S~5 and \S~5.1. &
see component predicate in \S~5.1 and \S~5.2.2 and zkSNARK wrapper in \S~5 and \S~5.1. \\

OIA\_03b &
For OIA\_03b, see Prepare relation in \S~5.2 and Show relation in \S~5.2. &
see Prepare relation in \S~5.2 and Show relation in \S~5.2. \\

OIA\_03c &
For OIA\_03c, see Prepare relation in \S~5.2 and Component predicate in \S~5.1 and \S~5.2.2. &
see Prepare relation in \S~5.2 and component predicate in \S~5.1 and \S~5.2.2. \\

OIA\_04 &
For OIA\_04, see Component predicate in \S~5.1 and \S~5.2.2 and Show relation in \S~5.2. &
see component predicate in \S~5.1 and \S~5.2.2 and Show relation in \S~5.2. \\

OIA\_05 &
For OIA\_05, see Component predicate in \S~5.1 and \S~5.2.2 and Security notes in \S~6. &
see component predicate in \S~5.1 and \S~5.2.2 and Security notes in \S~3. \\

OIA\_06 &
For OIA\_06, see Component predicate in \S~5.1 and \S~5.2.2. &
see component predicate in \S~5.1 and \S~5.2.2. \\

OIA\_07 &
For OIA\_07, see Component prepare batches in \S~5.2.1, Components zkSNARK wrapper in \S~5 and \S~5.1, and Component predicate in \S~5.1 and \S~5.2.2. &
see component prepare batches in \S~5.2.1, zkSNARK wrapper in \S~5 and \S~5.1, and component predicate in \S~5.1 and \S~5.2.2. \\

OIA\_08 &
For OIA\_08, see discussion in Security notes in \S~6 and Components zkSNARK wrapper in \S~5 and \S~5.1. &
see Security notes in \S~3 and zkSNARK wrapper in \S~5 and \S~5.1. \\

OIA\_09 &
For OIA\_09, see discussion in Security notes in \S~6 and Components zkSNARK wrapper in \S~5 and \S~5.1. &
see Security notes in \S~3 and zkSNARK wrapper in \S~5 and \S~5.1. \\

OIA\_10 &
For OIA\_10, see Component predicate in \S~5.1 and \S~5.2.2 and Show relation in \S~5.2. &
see component predicate in \S~5.1 and \S~5.2.2 and Show relation in \S~5.2. \\

OIA\_11 &
For OIA\_11, see Component predicate in \S~5.1 and \S~5.2.2 and Show relation in \S~5.2. &
see component predicate in \S~5.1 and \S~5.2.2 and Show relation in \S~5.2. \\

OIA\_12 &
For OIA\_12, see Prepare relation in \S~5.2.1 and discussion in Security notes in \S~6. &
see Prepare relation in \S~5.2.1 and Security notes in \S~3. \\

OIA\_13 &
For OIA\_13, see Prepare relation in \S~5.2.1 and discussion in Security notes in \S~6. &
see Prepare relation in \S~5.2.1 and Security notes in \S~3. \\

OIA\_14 &
For OIA\_14, see Prepare relation in \S~5.2.1 and discussion in Security notes in \S~6. &
see Prepare relation in \S~5.2.1 and Security notes in \S~3. \\

OIA\_15 &
For OIA\_15, see Prepare relation in \S~5.2.1 and discussion in Security notes in \S~6. &
see Prepare relation in \S~5.2.1 and Security notes in \S~3. \\

OIA\_16 &
For OIA\_16, see Security notes in \S~6 and Component predicate in \S~5.1 and \S~5.2.2. &
see Security notes in \S~3 and component predicate in \S~5.1 and \S~5.2.2. \\[1em]

\multicolumn{3}{l}{\textbf{Topic 10 — Issuance and Credential Handling (ISSU)}}\\

ISSU\_02 &
For ISSU\_02, see components SD-JWT/mDL wrapper in \S~5 and \S~5.1 and Components zkSNARK wrapper in \S~5 and \S~5.1. &
see SD-JWT/mDL wrapper in \S~5 and \S~5.1 and zkSNARK wrapper in \S~5 and \S~5.1. \\

ISSU\_07 &
For ISSU\_07, see components Prepare relation in \S~5.2.1 and prepareCommit in \S~5.2.1. &
see Prepare relation in \S~5.2.1 and \texttt{prepareCommit} in \S~5.2.1. \\

ISSU\_08 &
For ISSU\_08, see components Prepare relation in \S~5.2.1 and prepareCommit in \S~5.2.1. &
see Prepare relation in \S~5.2.1 and \texttt{prepareCommit} in \S~5.2.1. \\

ISSU\_09 &
For ISSU\_09, see components Prover's side discussion in \S~6 and Security notes in \S~6. &
see Prover-side discussion in \S~3 and Security notes in \S~3. \\

ISSU\_10 &
For ISSU\_10, see components Prepare relation in \S~5.2.1 and Prepare relation in \S~5.2. &
see Prepare relation in \S~5.2.1 and Prepare relation in \S~5.2. \\

ISSU\_12 &
For ISSU\_12, see components SD-JWT/mDL wrapper in \S~5 and \S~5.1 and Components zkSNARK wrapper in \S~5 and \S~5.1. &
see SD-JWT/mDL wrapper in \S~5 and \S~5.1 and zkSNARK wrapper in \S~5 and \S~5.1. \\

ISSU\_12a &
For ISSU\_12a, see components SD-JWT/mDL wrapper in \S~5 and \S~5.1 and Proof interface in \S~5.2. &
see SD-JWT/mDL wrapper in \S~5 and \S~5.1 and Proof interface in \S~5.2. \\

ISSU\_16 &
For ISSU\_16, see components SD-JWT/mDL wrapper in \S~5 and \S~5.1 and Interface in \S~5. &
see SD-JWT/mDL wrapper in \S~5 and \S~5.1 and Interface in \S~5. \\

ISSU\_27 &
For ISSU\_27, see components Show relation in \S~5.2.2 and Show relation in \S~5.2. &
see Show relation in \S~5.2.2 and Show relation in \S~5.2. \\

ISSU\_33 &
For ISSU\_33, see components SD-JWT/mDL wrapper in \S~5 and \S~5.1 and Component commitment in \S~5 and \S~5.2.1. &
see SD-JWT/mDL wrapper in \S~5 and \S~5.1 and component commitment in \S~5 and \S~5.2.1. \\

ISSU\_33a &
For ISSU\_33a, see components SD-JWT/mDL wrapper in \S~5 and \S~5.1 and Component commitment in \S~5 and \S~5.2.1. &
see SD-JWT/mDL wrapper in \S~5 and \S~5.1 and component commitment in \S~5 and \S~5.2.1. \\

ISSU\_33b &
For ISSU\_33b, see components prepareCommit in \S~5.2.1 and SD-JWT/mDL wrapper in \S~5 and \S~5.1. &
see \texttt{prepareCommit} in \S~5.2.1 and SD-JWT/mDL wrapper in \S~5 and \S~5.1. \\

ISSU\_35a &
For ISSU\_35a, see components prepareCommit in \S~5.2.1 and Prepare relation in \S~5.2. &
see \texttt{prepareCommit} in \S~5.2.1 and Prepare relation in \S~5.2. \\

ISSU\_37 &
For ISSU\_37, see components Component prepare batches in \S~5.2.1 and prepareBatch in \S~5.2.1. &
see component prepare batches in \S~5.2.1 and \texttt{prepareBatch} in \S~5.2.1. \\

ISSU\_38 &
For ISSU\_38, see components Show relation in \S~5.2.2 and Show relation in \S~5.2. &
see Show relation in \S~5.2.2 and Show relation in \S~5.2. \\

ISSU\_39 &
For ISSU\_39, see components prepareBatch in \S~5.2.1 and Prepare relation in \S~5.2. &
see \texttt{prepareBatch} in \S~5.2.1 and Prepare relation in \S~5.2. \\

ISSU\_44 &
For ISSU\_44, see components Component prepare batches in \S~5.2.1 and Show relation in \S~5.2. &
see component prepare batches in \S~5.2.1 and Show relation in \S~5.2. \\

ISSU\_58 &
For ISSU\_58, see components Show relation in \S~5.2.2 and Show relation in \S~5.2. &
see Show relation in \S~5.2.2 and Show relation in \S~5.2. \\

ISSU\_59 &
For ISSU\_59, see components Show relation in \S~5.2.2 and Show relation in \S~5.2. &
see Show relation in \S~5.2.2 and Show relation in \S~5.2. \\

ISSU\_60 &
For ISSU\_60, see components Component predicate in \S~5.1 and \S~5.2.2 and Proof interface in \S~5.2. &
see component predicate in \S~5.1 and \S~5.2.2 and Proof interface in \S~5.2. \\

ISSU\_61 &
For ISSU\_61, see components Component predicate in \S~5.1 and \S~5.2.2 and Proof interface in \S~5.2. &
see component predicate in \S~5.1 and \S~5.2.2 and Proof interface in \S~5.2. \\

ISSU\_62 &
For ISSU\_62, see components SD-JWT/mDL wrapper in \S~5 and \S~5.1 and Components zkSNARK wrapper in \S~5 and \S~5.1. &
see SD-JWT/mDL wrapper in \S~5 and \S~5.1 and zkSNARK wrapper in \S~5 and \S~5.1. \\

ISSU\_63 &
For ISSU\_63, see components prepareCommit in \S~5.2.1 and SD-JWT/mDL wrapper in \S~5 and \S~5.1. &
see \texttt{prepareCommit} in \S~5.2.1 and SD-JWT/mDL wrapper in \S~5 and \S~5.1. \\

ISSU\_64 &
For ISSU\_64, see components Proof interface in \S~5.2 and Proof interface in \S~5.2. &
see Proof interface in \S~5.2. \\[1em]

\multicolumn{3}{l}{\textbf{Topic 23 — PID and (Q)EAA issuance}}\\

Topic 23 &
23 PID issuance and (Q)EAA issuance. No HLRs, see Topic 10. &
covered by Topic 10 components in \S~5, \S~5.1, \S~5.2.1, \S~5.2.2. \\[1em]

\multicolumn{3}{l}{\textbf{Topic 47 — Protocols and interfaces for PID and (Q)EAA issuance}}\\

Topic 47 &
47 Protocols and interfaces for PID and (Q)EAA issuance and (non-)qualified. No HLRs, see Topic 10, 23. &
covered by Topic 10 / Topic 23 components in \S~5, \S~5.1, \S~5.2.1, \S~5.2.2. \\[1em]

\multicolumn{3}{l}{\textbf{Topic 53 — Zero-Knowledge Proofs (ZKP)}}\\

ZKP\_01 &
For ZKP\_01, see components Component predicate in \S~5.1 and \S~5.2.2 and Security notes in \S~6. &
see component predicate in \S~5.1 and \S~5.2.2 and Security notes in \S~3. \\

ZKP\_02 &
For ZKP\_02, see components Prepare relation in \S~5.2 and Component predicate in \S~5.1 and \S~5.2.2. &
see Prepare relation in \S~5.2 and component predicate in \S~5.1 and \S~5.2.2. \\

ZKP\_03 &
For ZKP\_03, see components Component commitment in \S~5 and \S~5.2.1 and Component linking in \S~5.2. &
see component commitment in \S~5 and \S~5.2.1 and component linking in \S~5.2. \\

ZKP\_04 &
For ZKP\_04, see components Component predicate in \S~5.1 and \S~5.2.2 and Show relation in \S~5.2. &
see component predicate in \S~5.1 and \S~5.2.2 and Show relation in \S~5.2. \\

ZKP\_05 &
For ZKP\_05, see components Prepare relation in \S~5.2 and Show relation in \S~5.2. &
see Prepare relation in \S~5.2 and Show relation in \S~5.2. \\

ZKP\_06 &
For ZKP\_06, see components Components zkSNARK wrapper in \S~5 and \S~5.1 and SD-JWT/mDL wrapper in \S~5 and \S~5.1. &
see zkSNARK wrapper in \S~5 and \S~5.1 and SD-JWT/mDL wrapper in \S~5 and \S~5.1. \\

ZKP\_07 &
For ZKP\_07, see components Prepare relation in \S~5.2.1 and Component commitment in \S~5 and \S~5.2.1 in Security notes in \S~6 and \S~5. &
see Prepare relation in \S~5.2.1 and component commitment in \S~5 and \S~5.2.1 and Security notes in \S~3. \\

ZKP\_08 &
For ZKP\_08, see components Backend modularity in \S~5.3 and Security notes in \S~6. &
see backend modularity in \S~5.3 and Security notes in \S~3. \\

ZKP\_09 &
For ZKP\_09, see components Show relation in \S~5.2.2 and Proof interface in \S~5.2. &
see Show relation in \S~5.2.2 and Proof interface in \S~5.2. \\

\end{longtable}
\end{landscape}





% Group 2
\clearpage
\begin{landscape}
\small
\begin{longtable}{p{3cm} p{10cm} p{7cm}}
\caption*{Requirements implementable by minor extension or modification of zkID components}\label{tab:groupB}\\
\toprule
\textbf{Annex 2 ID} &
\textbf{Requirement} &
\textbf{Coverage} \\
\midrule
\endfirsthead
\toprule
\textbf{Annex 2 ID} &
\textbf{Requirement} &
\textbf{Coverage} \\
\midrule
\endhead
\midrule
\multicolumn{3}{r}{\emph{continued on next page}}\\
\bottomrule
\endfoot
\bottomrule
\endlastfoot

\multicolumn{3}{l}{\textbf{Topic 6 — Relying Party authentication and User approval}}\\

RPA\_01 &
For RPA\_01, modify components Prepare relation in \S~5.2.1 and Prover's side discussion in \S~6. &
modify components Prepare relation in \S~5.2.1 and Prover's side discussion in \S~3. \\

RPA\_01a &
For RPA\_01a, modify components Show relation in \S~5.2 and Proof interface in \S~5.2. &
modify components Show relation in \S~5.2 and Proof interface in \S~5.2. \\

RPA\_02 &
For RPA\_02, modify components Show relation in \S~5.2 and Component predicate in \S~5.1 and \S~5.2.2. &
modify components Show relation in \S~5.2 and Component predicate in \S~5.1 and \S~5.2.2. \\

RPA\_02a &
For RPA\_02a, modify components Proof interface in \S~5.2 and Component linking in \S~5.2. &
modify components Proof interface in \S~5.2 and Component linking in \S~5.2. \\

RPA\_03 &
For RPA\_03, modify components Prepare relation in \S~5.2.1 and Show relation in \S~5.2.2. &
modify components Prepare relation in \S~5.2.1 and Show relation in \S~5.2.2. \\

RPA\_04 &
For RPA\_04, modify components Prover's side discussion in \S~6 and Security notes in \S~6. &
modify components Prover's side discussion in \S~3 and Security notes in \S~3. \\

RPA\_05 &
For RPA\_05, modify components Proof interface in \S~5.2 and Component linking in \S~5.2. &
modify components Proof interface in \S~5.2 and Component linking in \S~5.2. \\

RPA\_06 &
For RPA\_06, modify components Component predicate in \S~5.1 and \S~5.2.2 and Show relation in \S~5.2. &
modify components Component predicate in \S~5.1 and \S~5.2.2 and Show relation in \S~5.2. \\

RPA\_06a &
For RPA\_06a, modify components Show relation in \S~5.2 and Proof interface in \S~5.2. &
modify components Show relation in \S~5.2 and Proof interface in \S~5.2. \\

RPA\_07 &
For RPA\_07, modify components Show relation in \S~5.2.2 and Component predicate in \S~5.1 and \S~5.2.2. &
modify components Show relation in \S~5.2.2 and Component predicate in \S~5.1 and \S~5.2.2. \\

RPA\_07a &
For RPA\_07a, modify components Show relation in \S~5.2 and Proof interface in \S~5.2. &
modify components Show relation in \S~5.2 and Proof interface in \S~5.2. \\

RPA\_08 &
For RPA\_08, modify components Show relation in \S~5.2.2 and Show relation in \S~5.2. &
modify components Show relation in \S~5.2.2 and Show relation in \S~5.2. \\

RPA\_09 &
For RPA\_09, modify components Component predicate in \S~5.1 and \S~5.2.2 and Show relation in \S~5.2. &
modify components Component predicate in \S~5.1 and \S~5.2.2 and Show relation in \S~5.2. \\

RPA\_10 &
For RPA\_10, modify components Proof interface in \S~5.2 and Proof interface in \S~5.2. &
modify components Proof interface in \S~5.2 and Proof interface in \S~5.2. \\[1em]


\multicolumn{3}{l}{\textbf{Topic 11 — Pseudonyms}}\\

PA\_01 &
For PA\_01, modify components Component predicate in \S~5.1 and \S~5.2.2 and Prepare relation in \S~5.2. &
modify components Component predicate in \S~5.1 and \S~5.2.2 and Prepare relation in \S~5.2. \\

PA\_02 &
For PA\_02, modify components Show relation in \S~5.2 and Show relation in \S~5.2.2. &
modify components Show relation in \S~5.2 and Show relation in \S~5.2.2. \\

PA\_03 &
For PA\_03, modify components Proof interface in \S~5.2 and Proof interface in \S~5.2. &
modify components Proof interface in \S~5.2 and Proof interface in \S~5.2. \\

PA\_04 &
For PA\_04, modify components Prepare relation in \S~5.2 and Component predicate in \S~5.1 and \S~5.2.2. &
modify components Prepare relation in \S~5.2 and Component predicate in \S~5.1 and \S~5.2.2. \\

PA\_05 &
For PA\_05, modify components Proof interface in \S~5.2 and Proof interface in \S~5.2. &
modify components Proof interface in \S~5.2 and Proof interface in \S~5.2. \\

PA\_06 &
For PA\_06, modify components Show relation in \S~5.2 and Proof interface in \S~5.2. &
modify components Show relation in \S~5.2 and Proof interface in \S~5.2. \\

PA\_07 &
For PA\_07, modify components Proof interface in \S~5.2 and Proof interface in \S~5.2. &
modify components Proof interface in \S~5.2 and Proof interface in \S~5.2. \\

PA\_08 &
For PA\_08, modify components Proof interface in \S~5.2 and Proof interface in \S~5.2. &
modify components Proof interface in \S~5.2 and Proof interface in \S~5.2. \\

PA\_08a &
For PA\_08a, modify components Security notes in \S~6 and Proof interface in \S~5.2. &
modify components Security notes in \S~3 and Proof interface in \S~5.2. \\

PA\_09 &
For PA\_09, modify components Proof interface in \S~5.2 and Proof interface in \S~5.2. &
modify components Proof interface in \S~5.2 and Proof interface in \S~5.2. \\

PA\_10 &
For PA\_10, modify components Show relation in \S~5.2.2 and Security notes in \S~6. &
modify components Show relation in \S~5.2.2 and Security notes in \S~3. \\

PA\_11 &
For PA\_11, modify components Show relation in \S~5.2.2 and Security notes in \S~6. &
modify components Show relation in \S~5.2.2 and Security notes in \S~3. \\

PA\_12 &
For PA\_12, modify components Show relation in \S~5.2.2 and Show relation in \S~5.2.2. &
modify components Show relation in \S~5.2.2 and Show relation in \S~5.2.2. \\

PA\_13 &
For PA\_13, modify components Show relation in \S~5.2.2 and Proof interface in \S~5.2. &
modify components Show relation in \S~5.2.2 and Proof interface in \S~5.2. \\

PA\_14 &
For PA\_14, modify components Show relation in \S~5.2.2 and Security notes in \S~6. &
modify components Show relation in \S~5.2.2 and Security notes in \S~3. \\

PA\_15 &
For PA\_15, modify components Prepare relation in \S~5.2.1 and Security notes in \S~6. &
modify components Prepare relation in \S~5.2.1 and Security notes in \S~3. \\

PA\_16 &
For PA\_16, modify components Prepare relation in \S~5.2.1 and Component predicate in \S~5.1 and \S~5.2.2. &
modify components Prepare relation in \S~5.2.1 and Component predicate in \S~5.1 and \S~5.2.2. \\

PA\_17 &
For PA\_17, modify components Component prepare batches in \S~5.2.1 and Prepare relation in \S~5.2.1. &
modify components Component prepare batches in \S~5.2.1 and Prepare relation in \S~5.2.1. \\

PA\_18 &
For PA\_18, modify components prepareCommit in \S~5.2.1 and Security notes in \S~6. &
modify components \texttt{prepareCommit} in \S~5.2.1 and Security notes in \S~3. \\

PA\_19 &
For PA\_19, modify components Proof interface in \S~5.2 and Security notes in \S~6. &
modify components Proof interface in \S~5.2 and Security notes in \S~3. \\[1em]


\multicolumn{3}{l}{\textbf{Topic 17 — Identity matching}}\\

No HLRs &
Enable users to access existing online accounts or log in to cross-border public sector services using their PID via identity matching, even if PID attribute values do not exactly match those in the accounts. &
modify components Prepare relation in \S~5.2.1 and Security notes in \S~3. \\[2em]


\multicolumn{3}{l}{\textbf{Topic 18 — Combined presentations of attributes}}\\

ACP\_01 &
For ACP\_01, modify components Component predicate in \S~5.1 and \S~5.2.2 and Component linking in \S~5.2. &
modify components Component predicate in \S~5.1 and \S~5.2.2 and Component linking in \S~5.2. \\

ACP\_02 &
For ACP\_02, modify components Component commitment in \S~5 and \S~5.2.1 and Show relation in \S~5.2.2. &
modify components Component commitment in \S~5 and \S~5.2.1 and Show relation in \S~5.2.2. \\

ACP\_03 &
For ACP\_03, modify components Prepare relation in \S~5.2 and Show relation in \S~5.2. &
modify components Prepare relation in \S~5.2 and Show relation in \S~5.2. \\

ACP\_04 &
For ACP\_04, modify components Proof interface in \S~5.2 and Proof interface in \S~5.2. &
modify components Proof interface in \S~5.2 and Proof interface in \S~5.2. \\

ACP\_05 &
For ACP\_05, modify components Component commitment in \S~5 and Component predicate in \S~5.1 and \S~5.2.2. &
modify components Component commitment in \S~5 and Component predicate in \S~5.1 and \S~5.2.2. \\

ACP\_06 &
For ACP\_06, modify components prepareCommit in \S~5.2.1 and prepareBatch in \S~5.2.1. &
modify components \texttt{prepareCommit} in \S~5.2.1 and \texttt{prepareBatch} in \S~5.2.1. \\

ACP\_07 &
For ACP\_07, modify components Prepare relation in \S~5.2.1 and Component prepare batches in \S~5.2.1. &
modify components Prepare relation in \S~5.2.1 and Component prepare batches in \S~5.2.1. \\[1em]


\multicolumn{3}{l}{\textbf{Topic 20 — Strong User authentication for electronic payments}}\\

SUA\_01 &
For SUA\_01, modify components Show relation in \S~5.2.2 and Show relation in \S~5.2. &
modify components Show relation in \S~5.2.2 and Show relation in \S~5.2. \\

SUA\_02 &
For SUA\_02, modify components Component predicate in \S~5.1 and \S~5.2.2 and Show relation in \S~5.2. &
modify components Component predicate in \S~5.1 and \S~5.2.2 and Show relation in \S~5.2. \\

SUA\_03 &
For SUA\_03, modify components Show relation in \S~5.2 and Prepare relation in \S~5.2. &
modify components Show relation in \S~5.2 and Prepare relation in \S~5.2. \\

SUA\_04 &
For SUA\_04, modify components Show relation in \S~5.2.2 and Component predicate in \S~5.1 and \S~5.2.2. &
modify components Show relation in \S~5.2.2 and Component predicate in \S~5.1 and \S~5.2.2. \\

SUA\_05 &
For SUA\_05, modify components Security notes in \S~6 and Show relation in \S~5.2. &
modify components Security notes in \S~3 and Show relation in \S~5.2. \\

SUA\_06 &
For SUA\_06, modify components Component predicate in \S~5.1 and \S~5.2.2 and Proof interface in \S~5.2. &
modify components Component predicate in \S~5.1 and \S~5.2.2 and Proof interface in \S~5.2. \\[1em]


\multicolumn{3}{l}{\textbf{Topic 43 — Embedded disclosure policies}}\\

EDP\_01 &
For EDP\_01, modify components Component commitment in \S~5 and \S~5.2.1 and prepareCommit in \S~5.2.1. &
modify components Component commitment in \S~5 and \S~5.2.1 and \texttt{prepareCommit} in \S~5.2.1. \\

EDP\_02 &
For EDP\_02, modify components Component predicate in \S~5.1 and \S~5.2.2 and Proof interface in \S~5.2. &
modify components Component predicate in \S~5.1 and \S~5.2.2 and Proof interface in \S~5.2. \\

EDP\_03 &
For EDP\_03, modify components Prover's side discussion in \S~6 and Security notes in \S~6. &
modify components Prover's side discussion in \S~3 and Security notes in \S~3. \\

EDP\_05 &
For EDP\_05, modify components Proof interface in \S~5.2 and Proof interface in \S~5.2. &
modify components Proof interface in \S~5.2 and Proof interface in \S~5.2. \\

EDP\_06 &
For EDP\_06, modify components Component predicate in \S~5.1 and \S~5.2.2 and Show relation in \S~5.2. &
modify components Component predicate in \S~5.1 and \S~5.2.2 and Show relation in \S~5.2. \\

EDP\_07 &
For EDP\_07, modify components Component predicate in \S~5.1 and \S~5.2.2 and Show relation in \S~5.2. &
modify components Component predicate in \S~5.1 and \S~5.2.2 and Show relation in \S~5.2. \\

EDP\_09 &
For EDP\_09, modify components prepareCommit in \S~5.2.1 and SD-JWT/mDL wrapper in \S~5 and \S~5.1. &
modify components \texttt{prepareCommit} in \S~5.2.1 and SD-JWT/mDL wrapper in \S~5 and \S~5.1. \\

EDP\_10 &
For EDP\_10, modify components prepareCommit in \S~5.2.1 and Proof interface in \S~5.2. &
modify components \texttt{prepareCommit} in \S~5.2.1 and Proof interface in \S~5.2. \\

EDP\_11 &
For EDP\_11, modify components Security notes in \S~6 and Prepare relation in \S~5.2. &
modify components Security notes in \S~3 and Prepare relation in \S~5.2. \\[1em]


\multicolumn{3}{l}{\textbf{Topic 51 — PID or attestation deletion}}\\

PAD\_01 &
For PAD\_01, modify components Proof interface in \S~5.2 and Proof interface in \S~5.2. &
modify components Proof interface in \S~5.2 and Proof interface in \S~5.2. \\

PAD\_02 &
For PAD\_02, modify components prepareCommit in \S~5.2.1 and SD-JWT/mDL wrapper in \S~5 and \S~5.1. &
modify components \texttt{prepareCommit} in \S~5.2.1 and SD-JWT/mDL wrapper in \S~5 and \S~5.1. \\

PAD\_03 &
For PAD\_03, modify components Component predicate in \S~5.1 and \S~5.2.2 and Show relation in \S~5.2. &
modify components Component predicate in \S~5.1 and \S~5.2.2 and Show relation in \S~5.2. \\

PAD\_04 &
For PAD\_04, modify components Show relation in \S~5.2.2 and Show relation in \S~5.2. &
modify components Show relation in \S~5.2.2 and Show relation in \S~5.2. \\

PAD\_05 &
For PAD\_05, modify components Security notes in \S~6 and Component commitment in \S~5 and \S~5.2.1. &
modify components Security notes in \S~3 and Component commitment in \S~5 and \S~5.2.1. \\

PAD\_06 &
For PAD\_06, modify components Component prepare batches in \S~5.2.1 and Show relation in \S~5.2. &
modify components Component prepare batches in \S~5.2.1 and Show relation in \S~5.2. \\[1em]


\multicolumn{3}{l}{\textbf{Topic 52 — Relying Party intermediaries}}\\

RPI\_01 & 
For RPI\_01, modify components Proof interface in S5.2 and Interface in S3. &
modify components Proof interface in \S~5.2 and Interface in \S~3. \\

RPI\_02 &
For RPI\_02, N/A (empty). &
N/A. \\

RPI\_03 &
For RPI\_03, modify components Proof interface in S5.2 and Proof interface in S5.2. &
modify components Proof interface in \S~5.2 and Proof interface in \S~5.2. \\

RPI\_04 &
For RPI\_04, modify components Prepare relation in S5.2.1 and Security notes in S3. &
modify components Prepare relation in \S~5.2.1 and Security notes in \S~3. \\

RPI\_05 &
For RPI\_05, modify components Proof interface in S5.2 and Proof interface in S5.2. &
modify components Proof interface in \S~5.2 and Proof interface in \S~5.2. \\

RPI\_06 &
For RPI\_06, modify components Proof interface in S5.2 and Show relation in S5.2. &
modify components Proof interface in \S~5.2 and Show relation in \S~5.2. \\

RPI\_06a &
For RPI\_06a, modify components Proof interface in S5.2 and SD-JWT/mDL wrapper in S3 and S5.1. &
modify components Proof interface in \S~5.2 and SD-JWT/mDL wrapper in \S~3 and \S~5.1. \\

RPI\_07 &
For RPI\_07, modify components Proof interface in S5.2 and Show relation in S5.2. &
modify components Proof interface in \S~5.2 and Show relation in \S~5.2. \\

RPI\_07a &
For RPI\_07a, modify components Prepare relation in S5.2.1 and Show relation in S5.2.2. &
modify components Prepare relation in \S~5.2.1 and Show relation in \S~5.2.2. \\

RPI\_07b &
For RPI\_07b, modify components Interface in S3 and Show relation in S5.2.2. &
modify components Interface in \S~3 and Show relation in \S~5.2.2. \\

RPI\_08 &
For RPI\_08, modify components Proof interface in S5.2 and Security notes in S3. &
modify components Proof interface in \S~5.2 and Security notes in \S~3. \\

RPI\_09 &
For RPI\_09, modify components Prepare relation in S5.2.1 and Component linking in S5.2. &
modify components Prepare relation in \S~5.2.1 and Component linking in \S~5.2. \\

RPI\_10 &
For RPI\_10, modify components Security notes in S3 and Proof interface in S5.2. &
modify components Security notes in \S~3 and Proof interface in \S~5.2. \\

\end{longtable}
\end{landscape}



% Group 3
\clearpage
\begin{landscape}
\small
\begin{longtable}{p{3cm} p{10cm} p{7cm}}
\caption*{Requirements requiring integration with external systems or protocol adaptations}\label{tab:groupC}\\
\toprule
\textbf{Annex 2 Topic / ID} &
\textbf{Requirement} &
\textbf{Coverage} \\
\midrule
\endfirsthead
\toprule
\textbf{Annex 2 Topic / ID} &
\textbf{Requirement} &
\textbf{Coverage} \\
\midrule
\endhead
\midrule
\multicolumn{3}{r}{\emph{continued on next page}}\\
\bottomrule
\endfoot
\bottomrule
\endlastfoot

\multicolumn{3}{l}{\textbf{Topic 2 — Mobile Driving Licence (mDL) within the EUDI Wallet ecosystem}}\\

Topic 2 &
 &
\\

\multicolumn{3}{l}{\textbf{Topic 3 — PID Rulebook}}\\

Topic 3 &
 &
\\

\multicolumn{3}{l}{\textbf{Topic 4 — mDL Rulebook}}\\

Topic 4 &
 &
\\

\multicolumn{3}{l}{\textbf{Topic 7 — Attestation revocation and revocation checking}}\\

Topic 7 &
 &
\\

\multicolumn{3}{l}{\textbf{Topic 9 — Wallet Instance Attestation / Wallet Unit Attestation}}\\

Topic 9 &
 &
\\

\multicolumn{3}{l}{\textbf{Topic 10 — Issuance and Credential Handling (ISSU) (continued)}}\\

ISSU\_01     & & Protocol interoperability (OpenID4VCI profile); outside zkID components. \\
ISSU\_01a    & & Provider-side protocol interoperability (OpenID4VCI); outside zkID. \\
ISSU\_03     & & API-level interoperability (W3C Digital Credentials API); outside zkID. \\
ISSU\_04     & & Protocol feature: batch issuance in OpenID4VCI; outside zkID. \\
ISSU\_05     & & Wallet activation flow; wallet runtime responsibility, not zkID. \\
ISSU\_06     & & Post-issuance content check by wallet; outside zkID. \\
ISSU\_11     & & User-approval and display handled by wallet UX; outside zkID. \\
ISSU\_11b    & & Failure handling (delete and notify) by wallet; outside zkID. \\
ISSU\_12b    & & Per-credential keypair generation (device/HSM); key management, not zkID. \\
ISSU\_12c    & & Expiry alignment (PID $leq$ WUA expiry); lifecycle rule, not zkID. \\
ISSU\_12d    & & Expiry alignment for attestations (if revocation chaining); lifecycle rule, not zkID. \\
ISSU\_13–15  & & Rulebook compliance and OpenID4VCI support across Wallet/Provider; governance/protocol layer. \\
ISSU\_17–20  & & Device binding, high-assurance identification, trusted lists, published support; provider operations, not zkID. \\
ISSU\_21–23b & & Trusted lists and signed issuer metadata; wallet verifies; governance/protocol. \\
ISSU\_24–24a & & Wallet validates PID Provider access/registration before requesting; refuse if untrusted; wallet policy. \\
ISSU\_25–26  & & Attestation Providers: Rulebook compliance and OpenID4VCI support; provider-side. \\
ISSU\_27a–27c& & Attestation Provider subject/eligibility verification; provider-side. \\
ISSU\_28–32  & & Registration/trusted-lists/metadata publication; governance/protocol (ISSU\_31 empty). \\
ISSU\_34–34a & & Wallet checks provider authorisation/entitlement for requested type; block/warn if invalid. \\
ISSU\_35–35b & & Provider-side privacy (unique elements minimisation/discard); policy/operations. \\
ISSU\_36     & & Wallet-side minimisation/scope enforcement; wallet policy. \\
ISSU\_40–43  & & Anti-linkability methodswallet uses provider’s preference if supported; user impact minimal; batch semantics where applicable; protocol/UX. \\
ISSU\_45–50  & & Privacy-preserving handling during issuance flows; operational/protocol behaviours, not zkID components. \\
ISSU\_51–57  & & General issuance lifecycle policies beyond zkID scope (retries, scheduling, operational controls); wallet/provider operations. \\
ISSU\_65     & & Re-issuance binding continuity; protocol-layer control. \\
\multicolumn{3}{l}{\textbf{Topic 16 — Signing documents with a Wallet Unit}}\\

Topic 16 &
 &
\\

\multicolumn{3}{l}{\textbf{Topic 24 — User identification in proximity scenarios}}\\

Topic 24 &
 &
\\

\multicolumn{3}{l}{\textbf{Topic 25 — Unified definition and controlled vocabularies for attributes}}\\

Topic 25 &
 &
\\

\multicolumn{3}{l}{\textbf{Topic 26 — Catalogue of attestations}}\\

Topic 26 &
 &
\\

\multicolumn{3}{l}{\textbf{Topic 27 — Registration of PID Providers, Providers of QEAAs, PuB-EAAs, and non-qualified}}\\

Topic 27 &
 &
\\

\multicolumn{3}{l}{\textbf{Topic 30 — Interaction between Wallet Units}}\\

Topic 30 &
 &
\\

\multicolumn{3}{l}{\textbf{Topic 31 — Notification and publication of PID Provider / Wallet Provider / Attestation Provider trust status}}\\

Topic 31 &
 &
\\

\multicolumn{3}{l}{\textbf{Topic 33 — Wallet Unit backup and restore}}\\

Topic 33 &
 &
\\

\multicolumn{3}{l}{\textbf{Topic 35 — PID issuance service blueprint}}\\

Topic 35 &
 &
\\

\multicolumn{3}{l}{\textbf{Topic 37 — QES / Remote Signing — Technical Requirements}}\\

Topic 37 &
 &
\\

\multicolumn{3}{l}{\textbf{Topic 38 — Wallet Unit revocation}}\\

Topic 38 &
 &
\\

\multicolumn{3}{l}{\textbf{Topic 39 — Wallet-to-wallet technical topic}}\\

Topic 39 &
 &
\\

\multicolumn{3}{l}{\textbf{Topic 40 — Wallet Instance installation / activation / management}}\\

Topic 40 &
 &
\\

\multicolumn{3}{l}{\textbf{Topic 44 — Registration certificates for PID Providers, QEAAs, PuB-EAAs}}\\

Topic 44 &
 &
\\

\multicolumn{3}{l}{\textbf{Topic 48 — Blueprint for requesting data deletion to Relying Parties}}\\

Topic 48 &
 &
\\

\end{longtable}
\end{landscape}



% Group 4
\clearpage
\begin{landscape}
\small
\begin{longtable}{p{3cm} p{10cm} p{7cm}}
\caption*{Requirements dependent on product, user interface, or user-experience considerations}\label{tab:groupD}\\
\toprule
\textbf{Annex 2 Topic / ID} & \textbf{Requirement} & \textbf{Coverage}\\
\midrule
\endfirsthead
\toprule
\textbf{Annex 2 Topic / ID} & \textbf{Requirement} & \textbf{Coverage}\\
\midrule
\endhead
\midrule
\multicolumn{3}{r}{\emph{continued on next page}}\\
\bottomrule
\endfoot
\bottomrule
\endlastfoot

Topics 5, 8, 13, 14, 15, 21, 22, 36 &
There are no HLRs for this Topic. & \\

Topics 12, 32, 41, 45, 46 &
 & Refers to the attestation rulebook \\

Topics 24, 30 &
 & Refers to wallet instance requirements \\

Topic 28 &
 & Refers to PID rulebook \\

Topic 29 &
 & Refers to natural person PID \\

Topic 33 &
 & Refers to functional requirements of back up and restore function of wallet instance \\

Topic 42 &
... & ... \\

Topic 50 &
& Refers to compliance requirements of the wallet provider and wallet instance \\

Topic 54 &
... & ... \\

\end{longtable}
\end{landscape}
